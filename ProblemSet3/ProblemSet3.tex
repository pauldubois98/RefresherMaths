\documentclass[]{article}

%format
\usepackage[utf8]{inputenc}
\usepackage[T1]{fontenc}
\usepackage[english]{babel}
\usepackage[margin=2.5cm]{geometry}
%math
\usepackage{amsthm}
\usepackage{amsmath}
\usepackage{amsfonts}
\usepackage{amssymb}
\usepackage{stmaryrd}
\usepackage{nicefrac}
\usepackage{mathtools}
%others
\usepackage{hyperref}
\usepackage{graphicx}
\usepackage{enumitem}

%environments
\newtheorem{question}{Question}
%commands
%\newcommand{\name}[num]{definition}
\newcommand{\primes}{\mathbb{P}}
%\newcommand{\P}{\mathbb{P}}
\newcommand{\N}{\mathbb{N}}
\newcommand{\Z}{\mathbb{Z}}
\newcommand{\Q}{\mathbb{Q}}
\newcommand{\D}{\mathbb{D}}
\newcommand{\R}{\mathbb{R}}
\newcommand{\C}{\mathbb{C}}
\newcommand{\F}{\mathbb{F}}
\newcommand{\B}{\mathbb{B}}
\newcommand{\Norm}[2][]{\text{Norm}_{#1}(#2)}
\newcommand{\norm}[2][]{\text{Norm}_{#1}(#2)}
\newcommand{\inner}[2]{\left\langle #1,#2 \right\rangle}
\newcommand{\floor}[1]{\lfloor #1 \rfloor}
\newcommand{\ceil}[1]{\lceil #1 \rceil}
\newcommand{\abs}[1]{| #1 |}
\newcommand{\card}[1]{| #1 |}
\newcommand{\curt}[1]{\sqrt[3]{#1}}
\newcommand{\Ker}[1]{\text{Ker}(#1)}
\newcommand{\Image}[1]{\text{Im}(#1)}
\newcommand{\Trace}[1]{\text{Tr}(#1)}
\newcommand{\Det}[1]{\text{Det}(#1)}
\newcommand{\degree}[1]{\partial #1}
\newcommand{\Pow}[1]{\mathcal{P}(#1)}

%opening
\title{Problem Set 3}
\author{}
\date{Due 20\textsuperscript{th} September 2021}

\begin{document}

\maketitle

\begin{abstract}
	Only the questions with a star (*) are compulsory for submission;\\
	It is however \textit{strongly} advised to attempt all question.
\end{abstract}

\section{Metric/Norm Spaces}
\begin{question}
	(*) $X,\norm{\_}$ is a norm space; we define $d(x,y)=\norm{x-y}$, so that $X,d$ is a metric space.
	We have seen what the ball of radius $r$ centered at $x$ is in a metric space (i.e. using $d$).
	Define it in terms of then norm.
\end{question}

\begin{question}
	(*) We take $X=\R^2$, draw $B_1((0,0))$ (the unit ball) for the following norms:
	\begin{itemize}
		\item $\norm{(x,y)}=\abs{x}+\abs{y}$
		\item $\norm{(x,y)}=\sqrt{x^2+y^2}$
		\item $\norm{(x,y)}=\max(\abs{x},\abs{y})$
		\item $\norm{(x,y)}=0 \text{ if } (x,y)=(0,0) \text{, } \norm{(x,y)}=1 \text{ else}$
		\item $\norm{(x,y)}=0 \text{ if } (x,y)=(0,0) \text{, } \norm{(x,y)}=2 \text{ else}$
	\end{itemize}
\end{question}
\begin{question}
	For which values of $p\geq 0$ is $\norm{(x,y)}=\sqrt[p]{\abs{x}^p+\abs{y}^p}$ a norm for $\R^2$?
\end{question}
\begin{question}
	Find two different norms for $\R^n$ (give their definitions).
\end{question}



\section{Sequences Convergence}
\begin{question}
	Find the limit of the following sequences, and prove your claim carefully:
	\begin{itemize}
		\item (*) $x_n=\frac{1}{n^2}$
		\item $x_n=\frac{1}{n^3}$
		\item $x_n=\frac{1}{n^k}$ for some $k>0$
		\item (*) $x_n=\frac{1}{2^n}$
		\item $x_n=\frac{1}{3^n}$
		\item $x_n=\frac{1}{k^n}$ for some $k>1$
		\item (*) $x_n=\frac{3n^2+2n+1}{6n^2-n+2}$
		\item $x_n=\frac{2n+1}{6n^2+9n-5}$
		\item (hard) $x_n=\frac{n}{2^n}$
	\end{itemize}
\end{question}

\begin{question}
	$x_n  \to x$, $y_n \to y$;
	Prove the following:
	\begin{itemize}
		\item (*) $x_n+y_n \to x+y$
		\item $x_n*y_n \to x*y$
		\item if $y_n \neq 0$, and $y \neq 0$: $x_n/y_n \to x/y$
	\end{itemize}
\end{question}


\end{document}
