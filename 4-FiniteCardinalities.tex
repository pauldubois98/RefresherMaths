\chapter{Finite Cardinalities}

\begin{definition}[Cardinality] For finite sets:\\
	\emph{\underline{Intuitively}:} $\card{X} = n \in \N$ if there are $n$ elements in the set.\\
	\emph{\underline{Mathematically}:} $\card{X} = n \in \N$ if there is a bijection between $X$ and $\llbracket 1,n \rrbracket$.
\end{definition}
\begin{property}[Cardinality of Disjoints]
	$X,Y$ disjoint sets: $\card{X \cup Y} = \card{X} + \card{Y}$\\
	Extension: $X_1, \dots, X_n$ pairwise disjoint sets (i.e. $X_i \cap X_j = \emptyset \ \forall i \neq j$): $\card{\bigcup_{k=1}^n X_k} = \sum_{k=1}^{n} \card{X_k}$
\end{property}
\begin{proof}
	Shift bijection of $Y$ by $\card{Y}$; use induction.
\end{proof}
\begin{property}[Cardinality of Complement]
	$X \subseteq Y$: $\card{Y \setminus X} = \card{Y} - \card{X}$
\end{property}
\begin{proof}
	Use previous property with $X$ \& $Y \setminus X$ disjoint.
\end{proof}
\begin{property}[Cardinality of Cartesian Products]
	$X,Y$ sets: $\card{X \times Y} = \card{X} * \card{Y}$\\
	Extension: $X_1, \dots, X_n$ sets: $\card{\prod_{k=1}^n X_k} = \prod_{k=1}^{n} \card{X_k}$
	%proof: 
\end{property}
\begin{proof}
	$X \times \{y_k\}$  are all disjoint for $k \in \llbracket 1,\card{Y} \rrbracket$; use induction.
\end{proof}
\begin{property}[Cardinality of Sets of Functions]
	%Similar to Cartesian product:\\
	$\card{ \{f: X \to Y\} } = \card{Y}^{\card{X}}$
\end{property}
\begin{proof}
	Just count!
\end{proof}
\begin{property}[Cardinality of Sets of Injections]
	$\card{ \{f: X \to Y \mid f \text{ injective} \} } = \frac{\card{Y}!}{(\card{Y}-\card{X})!}$
\end{property}
\begin{proof}
	Count (without repetition).
\end{proof}
\begin{property}[Cardinality of Sets of Surjections]
	$\card{ \{f: X \to Y \mid f \text{ surjective} \} } = \card{Y}^{\card{X}} - \card{Y}*(\card{Y}-1)^{\card{X}}$
\end{property}
\begin{proof}
	All functions but the non surjective ones.
\end{proof}
\begin{property}[Cardinality of Sets of Bijections]
	$\card{ \{f: X \to Y \mid f \text{ bijective} \} } = \card{Y}! = \card{X}!$
\end{property}
\begin{proof}
	Bijection is an injection between two sets of the same size.
\end{proof}



