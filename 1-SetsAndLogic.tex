\chapter{Sets \& logic}

\section{Mathematical Objects \& Notations}
\subparagraph{Sets}
\begin{definition}[Sets]
	Unordered list of elements.
\end{definition}
\begin{notation}[Sets]
	$\in$, $\{ \text{True}, \text{False} \}$, $\{ a \mid condition \}$, $\{ a, b, c \dots \}$, $\emptyset$
\end{notation}
Need to be careful when defining set: some definitions are pathological.
\begin{remark}[Russell Paradox]
	Take $U = \{X \mid X \not\in X\}$.
	X in U => U not in U, U is a set, so not all sets are in U
	X not in U => X is a set
\end{remark}
\begin{notation}[Usual Sets]
	$\B$, $\N$, $\Z$, $\Q$, $\R$, $\C$, $\N^*$, $\R^+$...
\end{notation}

\subparagraph{Functions}
\begin{definition}[Functions]
	Assignment for a set to another.
\end{definition}
\begin{notation}[Function]
	$f: X \to Y$, $f(x)=blah$, $f: x \mapsto blah$.
\end{notation}
\begin{definition}[Predicate]
	Function to $\B$
\end{definition}
\begin{question}
	Which ones of these function are well-defined ?
	\begin{itemize}
		\item $f:k\in\{0,1,2,3,4\}\mapsto 24/k\in \N$
		\item $f:k\in \{1,2,3,4\}\mapsto 24/k\in \N$
		\item $f:k\in \{1,2,3,4,5\}\mapsto 24/k\in \N$
		\item $f:k\in \{1,2,3,4\}\mapsto k\in \{1,2\}$
		\item $f:k\in \{1,2,3,4\}\mapsto k\in \{1,2,3,4,5\}$
	\end{itemize}
\end{question}

\subparagraph{Quantifiers}
\begin{notation}[$\forall$]
	For all elements in set, e.g.: $\forall x \in \R, x^2 \geq 0$.
\end{notation}
\begin{notation}[$\exists$]
	There exists an element in set, e.g.: $\exists x \in \R \text{ s.t. } x^2 > 1$.
\end{notation}
\begin{notation}[$\exists !$]
	There exists a unique element in set, e.g.: $\exists ! x \in \R \text{ s.t. } x^2 \leq 0$.
\end{notation}
\begin{definition}[Subset / Inclusion]
	$X \subseteq Y$ if $\forall x \in X, x \in Y$
\end{definition}
\begin{definition}[Disjoint Sets]
	$X$ and $Y$ are disjoint if $\forall x \in X, x \not\in Y$ (or if $\forall y \in Y, y \not\in X$).
\end{definition}

\begin{definition}[Powerset]
	$\Pow{X} = \{ Y \mid Y \subseteq X \}$\\
	e.g.: $\Pow{\{1,2,3\}}=\{ \emptyset, \{1\},\{2\},\{3\}, \{1,2\},\{1,3\},\{2,3\}, \{1,2,3\} \}$
\end{definition}
\begin{definition}[Cartesian Product]
	$X \times Y = \{ (x,y) \mid x \in X, y \in Y \}$\\
	e.g.: $\{a,b\} \times \{1,2,3\} = \{ (a,1),(a,2),(a,3), (b,1),(b,2),(b,3) \}$\\
	Extension: $X_1 \times \dots \times X_n = \prod_{k=1}^n X_k$
\end{definition}



\section{Boolean algebra}
\subparagraph{Basic operators}
\begin{definition}[Conjonction]
	$x \land y = xy$
\end{definition}
\begin{definition}[Intersection]
	$X \cap Y = \{ z \mid (z \in X) \land (z \in Y) \}$
\end{definition}
\begin{remark}[Disjoint Sets and Intersection]
	Disjoint sets have empty intersection.
\end{remark}
\begin{definition}[Disjunction]
	$x \lor y = \min(x+y,1)$
\end{definition}
\begin{definition}[Union]
	$X \cup Y = \{ z \mid (z \in X) \lor (z \in Y) \}$
\end{definition}
\begin{definition}[Negation]
	$\lnot: 0,1 \mapsto 1,0$
\end{definition}
\begin{definition}[Set minus / Complement]
	$X \setminus Y = \{ x \in X \mid \lnot (x \in Y) \}$
\end{definition}
\begin{question}
	Selecting points outside a given region.
\end{question}
\subparagraph{Basic properties}
\begin{property}[Boolean algebra matching ordinary algebra]
	Same laws as ordinary algebra when one matches up $\lor$ with addition and $\land$ with multiplication.
	\begin{itemize}
		\item Associativity of $\lor$: $x \lor (y \lor z) = (x \lor y) \lor z$
		\item Associativity of $\land$: $x \land (y \land z) = (x \land y) \land z$
		\item Commutativity of $\lor$: $x \lor y  = y \lor x$
		\item Commutativity of $\land$: $x \land y  = y \land x$
		\item Distributivity of $\land$ over $\lor$:  $x \land (y \lor z) = (x \land y) \lor (x \land z)$
		\item $0$ is identity for $\lor$: $x \lor 0  = x$
		\item $1$ is identity for $\land$: $x \land 1  = x$
		\item $0$ is annihilator for $\land$: $x \land 0  = 0$
	\end{itemize}
\end{property}
\begin{property}[Boolean algebra specific properties]
	The following laws hold in Boolean algebra, but not in ordinary algebra: 
	\begin{itemize}
		\item Idempotence of $\lor$: $x \lor x = x$
		\item Idempotence of $\land$: $x \land x = x$
		\item Absorption of $\lor$ over $\land$: $x \lor (x \land y)  = x \land y$
		\item Absorption of $\land$ over $\lor$: $x \land (x \lor y)  = x \lor y$
		\item Distributivity of $\lor$ over $\land$:  $x \lor (y \land z) = (x \lor y) \land (x \lor z)$
		\item $1$ is annihilator for $\lor$: $x \lor 1 = 1$
	\end{itemize}
\end{property}
\begin{property}[De Morgan Laws]
	$\lnot (x \land y) = \lnot x \lor \lnot y$
	$\lnot (x \lor y) = \lnot x \land \lnot y$
\end{property}
\begin{proof}
	Truth-tables; prove De Morgan, others as exercise (or just believe me)
\end{proof}

\subparagraph{Other operators}
\begin{definition}[Exclusive Or]
	$x \oplus y$
\end{definition}
\begin{definition}[Implication]
	$x \implies y$
\end{definition}
\begin{property}[Implication and Inclusion]
	If $\forall x \in X, P_1(x) \implies P_2(x)$, then $\{ x \in X \mid P_1(x) \} \subset \{ x \in X \mid P_2(x) \}$.
\end{property}
\begin{proof}
	Trivial.
\end{proof}
\begin{definition}[If and only if]
	$x \iff y$
\end{definition}

\subparagraph{Negation of quantified propositions}
\begin{property}[Negation of $\forall$]
	$\mathrm{not}(\forall x\in X, P(x)) = \exists x\in X, \mathrm{not}(P(x))$
\end{property}
\begin{property}[Negation of $\exists$]
	$\mathrm{not}(\exists x\in X, P(x)) = \exists x\in X, \mathrm{not}(P(x))$
\end{property}
\begin{notation}[Quantifiers and the empty set]
	$\forall x \in \emptyset, \ \dots$ is true ;
	$\exists x \in \emptyset, \ \dots$ is false
\end{notation}



\section{Python}
=> use google colab'


