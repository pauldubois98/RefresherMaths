\documentclass[]{article}

%format
\usepackage[utf8]{inputenc}
\usepackage[T1]{fontenc}
\usepackage[english]{babel}
\usepackage[margin=2.5cm]{geometry}
%math
\usepackage{amsthm}
\usepackage{amsmath}
\usepackage{amsfonts}
\usepackage{amssymb}
\usepackage{stmaryrd}
\usepackage{nicefrac}
\usepackage{mathtools}
%others
\usepackage{hyperref}
\usepackage{graphicx}
\usepackage{enumitem}

%environments
\newtheorem{question}{Question}
%commands
%\newcommand{\name}[num]{definition}
\newcommand{\primes}{\mathbb{P}}
%\newcommand{\P}{\mathbb{P}}
\newcommand{\N}{\mathbb{N}}
\newcommand{\Z}{\mathbb{Z}}
\newcommand{\Q}{\mathbb{Q}}
\newcommand{\D}{\mathbb{D}}
\newcommand{\R}{\mathbb{R}}
\newcommand{\C}{\mathbb{C}}
\newcommand{\F}{\mathbb{F}}
\newcommand{\B}{\mathbb{B}}
\newcommand{\Norm}[2][]{\text{Norm}_{#1}(#2)}
\newcommand{\norm}[2][]{\text{Norm}_{#1}(#2)}
\newcommand{\inner}[2]{\left\langle #1,#2 \right\rangle}
\newcommand{\floor}[1]{\lfloor #1 \rfloor}
\newcommand{\ceil}[1]{\lceil #1 \rceil}
\newcommand{\abs}[1]{| #1 |}
\newcommand{\card}[1]{| #1 |}
\newcommand{\curt}[1]{\sqrt[3]{#1}}
\newcommand{\Ker}[1]{\text{Ker}(#1)}
\newcommand{\Image}[1]{\text{Im}(#1)}
\newcommand{\Trace}[1]{\text{Tr}(#1)}
\newcommand{\Det}[1]{\text{Det}(#1)}
\newcommand{\degree}[1]{\partial #1}
\newcommand{\Pow}[1]{\mathcal{P}(#1)}

%opening
\title{Problem Set 4}
\author{}
\date{Due 24\textsuperscript{nd} September 2021}

\begin{document}

\maketitle

\begin{abstract}
	Only the questions with a star (*) are compulsory for submission;\\
	It is however \textit{strongly} advised to attempt all question.
\end{abstract}

\section{Sequences}
\begin{question}
	$X,\norm{\_}$ is a norm space; $(x_n)$ is a sequence in $X$.
	Define what is meant by $x_n \to x$ in this case (remember we defined it in class for $X,d$ a metric space, and we also saw how to define a metric space from a norm space).
\end{question}

\begin{question}(Arithmetic Sequence)
	$(x_n)$ is defined by iteration as follows: $x_0=b$, $x_{n+1}=a+x_n$.\\
	a*) Prove that the explicit formula for $x_n$ is $x_n=b+a*n$.\\
	b*) State and prove the behavior as $n \to +\infty$ (split in cases for different values of $a$ or $b$ if needed).
\end{question}
\begin{question}(Geometric  Sequence)
	$(x_n)$ is defined by iteration as follows: $x_0=b$, $x_{n+1}=a*x_n$.\\
	a*) Prove that the explicit formula for $x_n$ is $x_n=b*a^n$.\\
	b*) State and prove the behavior as $n \to +\infty$ (split in cases for different values of $a$ or $b$ if needed).
\end{question}

\begin{question}
	The Syracuse sequence is defined by the recurrence relation as follows:
	\begin{itemize}
		\item $x_{n+1}=\frac{x_n}{2}$ if $x_n$ is even
		\item $x_{n+1}=3*x_n+1$ if $x_n$ is odd
	\end{itemize}
	We start with a natural number (note that the sequence takes only integral values).\\
	a) Conjecture how many iterations it takes to reach 1 starting from 7,8,15,16.\\
	b) Calculate how many iterations it takes to reach 1 starting from 7,8,15,16.\\
	c) Was your first intuition correct?\\
	(You may want to use a calculator to speed up calculations)
\end{question}

\begin{question}
	(*)	Let $f(x) = \frac{2x^2-3x+9}{4x-5}$\\
	Find the asymptote as $x \to \pm\infty$; Find the singularity of $f$.\\
	Draw a graph of $f$ to the best of your knowledge, using the above information.
\end{question}

\begin{question}
	(*) Let $x_{n+1}=\frac{1}{2}x_n+1$, $x_0=0$.\\
	Show that the general term of this sequence is given by $x_n=2-\frac{2}{2^n}$.\\
	Deduce the limit of the sequence.
\end{question}

\begin{question}
	Find a set $X$ such that $\forall x,y \in X, d(x,y)<\text{Diam}(X)$ (i.e. $\text{Diam}(X)=\sup(\{d(x,y) \mid x,y \in X\})$ but $\text{Diam}(X) \neq \max(\{d(x,y) \mid x,y \in X\})$).\\
	($X,d$ can be a metric space of your choice, but I advise $X=\R$, $d(x,y)=\abs{x-y}$ to begin with)
\end{question}


\end{document}
