\chapter{Spaces}
Mathematical Space: Object based on a set with more structure.
\section{Metric Space}
A metric space is a set $X$ together with a metric distance $d: X \times X \to \R^+$.\\
$d$ is a metric if it satisfies the following axioms:
\begin{itemize}
	\item Non-degenerative: $d(x,y)=0 \iff x=y$
	\item Symmetric: $d(x,y) = d(y,x)$
	\item Triangle inequality: $d(x,z) \leq d(x,y) + d(y,z)$
\end{itemize}
\begin{question}
	Which of the following are metric spaces?
	\begin{itemize}
		\item $X=\N$, $d(x,y)=1$
		\item $X=\N$, $d(x,y)=0$ if $x=y$, $d(x,y)=1$ else
		\item $X=\R$, $d(x,y) = \abs{x-y}$
		\item $X=\R$, $d(x,y) = (x-y)^2$
	\end{itemize}
\end{question}

\section{Norm Space}
A norm space is a set $X$ together with a norm $\abs{\_}: X \to \R^+$.\\
$\abs{\_}$ is a norm if it satisfies the following axioms:
\begin{itemize}
	\item Non-degenerative: $\norm{x}=0 \iff x=0$
	\item Homogeneity: $\norm{\lambda x} = \abs{\lambda} \norm{x} \qquad \lambda \in \R$
	\item Triangle inequality: $\norm{x+y} \leq \norm{x} + \norm{y}$
\end{itemize}
\begin{question}
	Which of the following are norm spaces?
	\begin{itemize}
		\item $X=\R^2$, $\norm{(x,y)}=\sqrt{x^2+y^2}$
		\item $X=\R^2$, $\norm{(x,y)}=\abs{x}+\abs{y}$
		\item $X=\R^2$, $\norm{(x,y)}=\max(\abs{x}+\abs{y})$
		\item $X=\R^n$, $\norm{(x_1,x_2,\dots,x_n)}=\sqrt{x_1^2+_2^2+\dots+x_n^2}$
	\end{itemize}
\end{question}
Hint: to show triangle inequality, search for "Minkowski inequality"
\begin{property}[Norm Implies Metric]
	Letting $d(x,y) = \norm{x-y}$.
\end{property}

\section{Inner Product Space}
An inner product space is a set $X$ together with an inner product $\inner{\_}{\_}: X \times X \to \C$.\\
$\inner{\_}{\_}$ is an inner product if it satisfies the following axioms:
\begin{itemize}
	\item Linear (in $1^{\text{st}}$ argument): $\inner{\lambda x}{y} = \lambda \inner{x}{y} \quad \lambda \in \C$ and $\inner{x+x'}{y} = \inner{x}{y} +\inner{x'}{y}$
	\item Conjugate symmetry: $\abs{x+y} \leq \abs{x} + \abs{y}$
	\item Positive definiteness $\inner{x}{x}>0 \ \forall x \neq 0$
	\item \textit{(implied)} Non-degenerative: $\inner{x}{0}=0$ and $\inner{0}{x}=0$
	\item \textit{(implied)} Conjugate linear (in $2^{\text{nd}}$ argument): $\inner{x}{\lambda y} = \bar{\lambda} \inner{x}{y} \quad \lambda \in \C$ and $\inner{x}{y+y'} = \inner{x}{y} +\inner{x}{y'}$
\end{itemize}
\begin{property}[Inner Product implies Norm]
	Letting $\abs{x} = \sqrt{\inner{x}{x}}$.
\end{property}

\begin{property}[Cauchy-Schwarz inequality]
	$\inner{x}{y}^2 \leq \inner{x}{x} \inner{y}{y}$
\end{property}
\begin{proof}
	Let $P(\lambda)=\inner{x+\lambda y}{x+\lambda y}$.
	This polynomial is never negative, so its discriminant must be non-positive.
	Deduce the inequality from $\Delta \geq 0$.
\end{proof}
\begin{definition}[Orthogonal / Normal]
	$x,y \text{ orthogonal } \iff x \perp y \iff \inner{x}{y}=0$
\end{definition}
\begin{property}[Pythagoras Theorem]
	$x \perp y \implies \abs{x+y}^2 = \abs{x}^2 + \abs{y}^2$
\end{property}
\begin{property}[Parallelogram Identity]
	$\abs{x+y}^2 +\abs{x-y}^2 = 2 (\abs{x}^2 + \abs{y}^2)$
\end{property}
\begin{property}[Polarization Identity]
	$4 \inner{x}{y} = \abs{x+y}^2 - \abs{x-y}^2 + i(\abs{x+iy}^2 - \abs{x-iy}^2)$
\end{property}

\begin{question}
	Draw ball of radius one in $\R^2$ for the following norms: $\abs{\_}_1$, $\abs{\_}_2$, $\abs{\_}_3$, $\abs{\_}_{\infty}$.
\end{question}



\section{Openness}
Here, we work over $(X,d)$, a metric space.
\begin{definition}[Open Ball]
	$B(x_0,r) = \{ x \in X \mid d(x,x_0)<r \}$
\end{definition}
\begin{definition}[Closed Ball]
	$\overline{B}(x_0,r) = \{ x \in X \mid d(x,x_0) \leq r \}$
\end{definition}
\begin{definition}[Open Set]
	$U \text{ is open } \iff \forall x \in U, \exists \epsilon>0 \text{ s.t. } B(x_0,\epsilon) \subseteq U$
\end{definition}
\begin{definition}[Closed Set]
	$C \text{ is closed } \iff X \setminus C \text{ is open}$
\end{definition}
\begin{property}
	Open balls are open.
\end{property}
\begin{proof}
	Use triangle inequality \& draw scheme
\end{proof}
\begin{property}
	Closed balls are closed.
\end{property}
\begin{proof}
	Use triangle inequality \& draw scheme
\end{proof}

