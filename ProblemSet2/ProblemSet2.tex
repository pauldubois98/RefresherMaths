\documentclass[]{article}

%format
\usepackage[utf8]{inputenc}
\usepackage[T1]{fontenc}
\usepackage[english]{babel}
\usepackage[margin=2.5cm]{geometry}
%math
\usepackage{amsthm}
\usepackage{amsmath}
\usepackage{amsfonts}
\usepackage{amssymb}
\usepackage{stmaryrd}
\usepackage{nicefrac}
\usepackage{mathtools}
%others
\usepackage{hyperref}
\usepackage{graphicx}
\usepackage{enumitem}

%environments
\newtheorem{question}{Question}
%commands
%\newcommand{\name}[num]{definition}
\newcommand{\primes}{\mathbb{P}}
%\newcommand{\P}{\mathbb{P}}
\newcommand{\N}{\mathbb{N}}
\newcommand{\Z}{\mathbb{Z}}
\newcommand{\Q}{\mathbb{Q}}
\newcommand{\D}{\mathbb{D}}
\newcommand{\R}{\mathbb{R}}
\newcommand{\C}{\mathbb{C}}
\newcommand{\F}{\mathbb{F}}
\newcommand{\B}{\mathbb{B}}
\newcommand{\Norm}[2][]{\text{Norm}_{#1}(#2)}
\newcommand{\inner}[2]{\left\langle #1,#2 \right\rangle}
\newcommand{\floor}[1]{\lfloor #1 \rfloor}
\newcommand{\ceil}[1]{\lceil #1 \rceil}
\newcommand{\abs}[1]{| #1 |}
\newcommand{\card}[1]{| #1 |}
\newcommand{\curt}[1]{\sqrt[3]{#1}}
\newcommand{\Ker}[1]{\text{Ker}(#1)}
\newcommand{\Image}[1]{\text{Im}(#1)}
\newcommand{\Trace}[1]{\text{Tr}(#1)}
\newcommand{\Det}[1]{\text{Det}(#1)}
\newcommand{\degree}[1]{\partial #1}
\newcommand{\Pow}[1]{\mathcal{P}(#1)}

%opening
\title{Problem Set 2}
\author{}
\date{Due 14\textsuperscript{th} September 2021}

\begin{document}

\maketitle

\begin{abstract}
	Only the questions with a star (*) are compulsory for submission;\\
	It is however \textit{strongly} advised to attempt all question.
\end{abstract}

\section{Functions Properties}
\begin{question}
	Show that the composition preserve injectivity/surjectivity/bijectivity/invertibility:
	$f: X \to Y, \ g: Y \to Z \text{ injectives } \implies f \circ g \text{ is injective}$\\
	$f: X \to Y, \ g: Y \to Z \text{ surjectives } \implies f \circ g \text{ is surjective}$\\
	$f: X \to Y, \ g: Y \to Z \text{ bijectives/invertibles } \implies f \circ g \text{ is bijective/invertible}$
\end{question}
\begin{question}
	(*) An injection between two sets of the same size is bijective.
\end{question}

\section{Finite Cardinalities}

\begin{question}[Counting the number of functions between two finite sets]
	Let $X$ and $Y$ be two non-empty finite sets. We want to count the number of functions in $Y^X$.
	\begin{enumerate}[label=\alph*.]
		\item Let us assume that $\card{X} = n$ with $X = \{x_1,...,x_n\}$. Prove that the function
		\begin{equation*}
			\Phi : \begin{array}{lcl}
				Y^X&\longrightarrow&Y^n\\
				f&\longmapsto& (f(x_1),...,f(x_n))
			\end{array}
		\end{equation*}
		is a bijection.
		\item Deduce the value of $\card{Y^X}$.
		\item Let $n\in \N^*$. Let us consider the set $\mathfrak{S}_n\subset \llbracket 1;n\rrbracket^{\llbracket 1;n\rrbracket}$ containing the bijections from $\llbracket 1;n\rrbracket$ to itself. Prove that the sequence $(\card{(\mathfrak{S}_n)})_{n\in \N^*}$ is defined by the recurrence relation
		\begin{equation*}
			\left\{\begin{array}{l}
				\card{\mathfrak{S}_1} = 1\\
				\forall n\in \N,~\card{\mathfrak{S}_{n+1}} = (n+1)\card{\mathfrak{S}_n}
			\end{array}\right.
		\end{equation*}
		\textit{(hint : we can use the bijections $\forall k\in \llbracket 1;n\rrbracket,~ g_k:\llbracket 1;n\rrbracket \setminus \{k\}\rightarrow \llbracket 1;n-1\rrbracket$)}
		\item The cardinal of $\mathfrak{S}_n$ is the factorial of $n$, denoted as $n!$. Write a function returning the value of the factorial for a given $n\in \N$ (by convention $0! = 1$).
	\end{enumerate}
\end{question}

\begin{question}[Counting the number of sub-parts]
	(*) We study the function $(n,p)\in \N^2\mapsto \binom{n}{k}\in \N$ the binomial coefficient, which is the number of subsets containing $p$ elements in a set containing $n$ elements.
	\begin{enumerate}[label=\alph*.]
		\item Prove that $\forall n,p\geq 1,~\displaystyle \binom{n}{p} = \binom{n-1}{p-1}+\binom{n-1}{p}$.
		\item Deduce from this recurrence equation that
		\begin{equation*}
			\forall n\in \N,~\forall p\leq n,~\binom{n}{p} = \frac{n!}{(n-p)!p!}
		\end{equation*}
		\item Prove the formula
		\begin{equation*}
			\forall n\in \N, \sum_{k=0}^n \binom{n}{k} = 2^n
		\end{equation*}
		\item Derive the value of $\card{\mathcal{P}(\llbracket 1;n\rrbracket)}$ (power set).
	\end{enumerate}
\end{question}

\section{Infinite Cardinalities}
\begin{question}
	(*) Find an explicit bijection between $\left[ a,b \right]$ and $\left[ c,d \right]$
\end{question}
\begin{question}
	(*) Show that there is a bijection between $\left[ 0,1 \right[$ and $\{0,1\}^\N$
\end{question}
\begin{question}
	(*) Let $\mathcal{A} = \{a,b,c,\dots,z\}$ be the set of letters in the alphabet.
	Show explicitly that $\card{\mathcal{A} \cup \N}=\card{\N}$.
\end{question}
\begin{question}
	Let $\C$ be the set of complex numbers. Compare $\R$ and $\C$.
\end{question}
\begin{question}
	Let $\primes$ be the set of prime numbers. Compare $\primes$ with the usual sets (in particular with $\N$, $\Z$, $\Q$, $\R$).
\end{question}

\end{document}
