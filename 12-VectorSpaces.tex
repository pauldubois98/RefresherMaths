\chapter{Vector Spaces}
\section{Axioms}
\begin{definition}[Vector Space]
	$\F$ a field (usually $\R$ or $\C$).
	$V$ is a vector field over $\F$ if:
	\begin{itemize}
		\item [$V$ has addition] $\forall v,w \in V: v+w \in V$
		\item [$V$ has multiplication by a scalar] $\forall v \in V, k \in \F: k.v \in V$
	\end{itemize}
	Such that:
	\begin{itemize}
		\item $\forall v \in V, k,l \in \F: (kl).v = k.(l.v)$
		\item $\forall v \in V, k,l \in \F: (k+l).v = k.v + l.v$
		\item $\forall v,w \in V, k \in \F: k.(v+w) = k.v + k.w$
		\item $\forall v \in V: 1.v=v$
		\item $\forall v \in V: 0.v=\textbf{0}$
		\item $\forall k \in \F: k.\textbf{0}=\textbf{0}$
	\end{itemize}
\end{definition}
\begin{example}
	\begin{itemize}
		\item $0$ over $\R$
		\item $\R$ over $\R$
		\item $\C$ over $\R$
		\item $\C$ over $\C$
		\item $\R^n$ over $\R$
		\item $\R\left[x\right]$ over $\R$ (real polynomials)
		\item $\R^{\N}$ over $\R$ (real sequences)
		\item $\R^{\R}$ over $\R$ (real functions)
		\item $V_1 \times V_2$ if $V_1$ and $V_2$ are vector spaces over the same field $\F$
	\end{itemize}
\end{example}

\begin{definition}[Vector Sub-space]
	$W \subseteq V$ is a vector subspace of $V$ if it is a vector space on its own.
	Need to check:
	\begin{itemize}
		\item Closed under addition: $w,w' \in W \implies w+w' \in W$
		\item Multiplication by a scalar: $w \in W, k \in \F \implies k.w \in W$
		\item Contains the null vector: $\textbf{0} \in W$
	\end{itemize}
\end{definition}
\begin{example}
	\begin{itemize}
		\item $0$ is a vector sub-space of $\R$
		\item $\R$ is a vector sub-space of $\C$ over $\R$
		\item $\R_d\left[x\right]$ is a vector sub-space of $\R\left[x\right]$ over $\R$ (real polynomials of degree $d$)
		\item $C\left[ \R\right]$ is a vector sub-space of $\R^{\R}$ over $\R$ (real functions)
	\end{itemize}
\end{example}
\begin{proposition}
	$W_1, W_2 \text{ subspaces of } V \implies W_1 \cap W_2 \text{ subspace of } V$
\end{proposition}
\begin{proof}
	in problem set
\end{proof}

\begin{property}[Direct product of vector spaces are vector spaces]
	$V_1, V_2 \text{ vector spaces } \implies V_1 \times V_2 \text{ is a vector space}$\\
	$(v_1,v_2),(v_1',v_2') \in V_1 \times V_2, k \in \F$:
	\begin{itemize}
		\item $(v_1,v_2)+(v_1',v_2')=(v_1+v_1',v_2+v_2')$
		\item $k.(v_1,v_2)=(k.v_1,k.v_2)$
	\end{itemize}
\end{property}

\begin{definition}[Linear combination]
	$A \subset V \text{ vector space}$: $x$ is a linear combination of $A$ iff $\exists k \in \N \text{ s.t. } \exists k_1,\dots,k_n \in \F, x_1,\dots,x_n \in A \text{ s.t. } x=\sum_{i=1}^{n}k_i.x_i$
\end{definition}
\begin{definition}[Span]
	$A \subset V \text{ vector space}$: $\Span{A}$ is the set of all vector that can be expressed as a linear combination of $A$, i.e. $\Span{A}=\{ \sum_{i=1}^{n}k_i.x_i \mid n\in \N, k_i \in \F, x_i\in A \}$
\end{definition}
\begin{example}
	\begin{itemize}
		\item $\Span{\{\textbf{1}\}}=\R$
		\item $\Span{\{\textbf{0}\}}=\{\textbf{0}\}$
		\item $\Span{\{\textbf{1},\textbf{i}\}}=\C$
	\end{itemize}
\end{example}
\begin{property}
	$\Span{A}$ is the smallest vector space containing $A$
\end{property}
\begin{proof}
	any smaller set containing $A$ is not closed under addition/multiplication by scalar
\end{proof}
\begin{exercise}
	Which of these can be seen as vector spaces?
	\begin{itemize}
		\item $\R^{\N}$
		\item $\{ (x,y) \mid x^2+y^2 \leq 1 \}$ "unit disk"
		\item $\{ (x,y) \mid x+y=0 \}$
		\item $\{ (x,y) \mid x+y=1 \}$
		\item $\{ f \in C\left[ a,b \right] \mid f(a)=f(b) \}$
	\end{itemize}
\end{exercise}



\section{Dimension \& Basis}
\begin{definition}[Linear Independence]
	$x_1,\dots,x_n$ are linearly independent (LI) if $\forall k_1,\dots,k_n \in \F, \sum_{i=1}^{n}k_i.x_i = 0 \implies \forall 1 \leq i \leq n, k_i=0$
\end{definition}
\begin{example}
	\begin{itemize}
		\item $(1,0,0)$ \& $(0,1,0)$ in $\R^3$
		\item $1$ \& $i$ in $\C$
		\item $(1,0,1)$, $(5,0,1)$ \& $(1,3,0)$ in $\R^3$
	\end{itemize}
\end{example}
\begin{definition}[Basis of a Vector Space]
	$V$ has basis $B \subset V$ if $\Span{B}=V$ and $B$ is LI.
\end{definition}
\begin{example}
	\begin{itemize}
		\item $\{(1,0,0), (0,1,0), (0,0,1)\}$ in $\R^3$
		\item $\{1,i\}$ in $\C$
	\end{itemize}
\end{example}
The LI property tends to make the basis "small", while the spanning property tends to make it "large".\\
A basis is a largest LI set, or a smaller spanning set.
\begin{property}[Basis always exist]
	Any vector space $V$ has a basis
\end{property}
\begin{proof}
	Either remove from a spanning set (e.g. $V$ spans itself), or (in finite dimensions), add elements to LI set until the set spans all of $V$.
\end{proof}
\begin{property}[All Basis have the Same Number of Elements]
	If $B$ and $B'$ are both basis of $V$ (i.e. span $V$ and are LI), then $\card{B}=\card{B'}$
\end{property}
\begin{proof}
	Technical, omitted
\end{proof}

\begin{definition}[Dimension]
	The dimension of a vector space is the cardinal of a basis (note it ay be finite of infinite)
\end{definition}
\begin{example}
	\begin{itemize}
		\item $\R^3$ has dimension 3 (finite)
		\item $\R\left[ x \right]$ has infinite dimension
		\item $\dim_\R(\C)=2$
		\item $\dim_\C(\C)=1$
		\item $\dim(\R^n)=n$
		\item $\dim(\{\textbf{0}\})=n$
		
	\end{itemize}
\end{example}
\begin{property}[Algebra of Dimensions]
	\begin{itemize}
		\item $\dim(V \times V')=\dim(V)+\dim(V')$
		\item $\dim(V + V')=\dim(V)+\dim(V')-\dim(V \cap V')$
	\end{itemize}
\end{property}




