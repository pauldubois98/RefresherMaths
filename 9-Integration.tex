\chapter{Integration}

Integral of $f$: area under a curve of $f$ (draw scheme).

\begin{proof}
	Take $f$ continuous;
	let $A(x)$ be the area under the curve of $f$, from $0$ to $x$.
	
	Then $A(x+h) - A(x) = f(x).h + \epsilon(h)$; as $h \to 0$, $\epsilon(h) \to 0$ by continuity of $f$.\\
	Then get $f(x) = \frac{A(x+h) -A(x)}{h}$ as $h \to 0$.
	Thus, $A'(x) = f(x)$.
\end{proof}
\begin{notation}
	Integral from $a$ to $b$ of $f$ is $\int_{a}^{b} f(x) dx$
\end{notation}
\begin{theorem}[Fundamental Theorem of Calculus]
	If $F(x) = \int_{a}^{x} f(t) dt$, then $F$ is uniformly continuous and differentiable, with derivative $F' = f$
\end{theorem}
\begin{corollary}
	$\int_{a}^{b} f(x) dx = F(b) - F(a)$ where $F$ is an anti-derivative of $f$ (i.e. $F' = f$\footnote{Anti-derivative are \textbf{not} unique (can add a constant).})
\end{corollary}

Integrals may also be approximated via partial sums; this is how computers calculate integrals (draw picture).

Integration is the "inverse" of differentiation: $\int f = F + C$ where $C \in \R$ and $F' = f$.\\
So $\int f' = f + C$ and $\left( \int f \right)' = f$

Note that not all functions are integrable in terms of elementary functions (e.g.: $\frac{\sin(x)}{x}$).
Note that not all functions are integrable in terms of area under the curve either (e.g.: $f(x)=0 \text{ if } x \in Q, f(x)=1 \text{ if } x \not\in \Q$).
However, "most" usual functions are integrable in terms of area under the curve (any continuous or monotone function is, so usually do not worry about it in applied maths).

Note that integrable does \textbf{not} imply differentiable/continuous (e.g. floor function); and differentiable does \textbf{not} imply anti-derivative exists in terms of elementary functions (e.g. $\frac{\sin(x)}{x}$).
[draw diagram of implications: integrable area; integrable anti-derivative; continuous; differentiable]
