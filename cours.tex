\documentclass[11pt,a4paper]{report}

%format
\usepackage[utf8]{inputenc}
\usepackage[T1]{fontenc}
\usepackage[english]{babel}
\usepackage[margin=2.5cm]{geometry}
%math
\usepackage{amsthm}
\usepackage{amsmath}
\usepackage{amsfonts}
\usepackage{amssymb}
\usepackage{stmaryrd}
\usepackage{nicefrac}
\usepackage{mathtools}
%others
\usepackage{hyperref}
\usepackage{graphicx}

%environments
\newtheorem*{remark}{Remark}
\newtheorem*{notation}{Notation}
\newtheorem*{definition}{Definition}
\newtheorem*{question}{Question}
\newtheorem*{proposition}{Proposition}
\newtheorem*{property}{Property}
\newtheorem{lemma}{Lemma}[section]
\newtheorem{theorem}{Theorem}[section]
\newtheorem{corollary}{Corollary}[section]
\newtheorem{conjecture}{Conjecture}[section]
%commands
%\newcommand{\name}[num]{definition}
\newcommand{\primes}{\mathbb{P}}
%\newcommand{\P}{\mathbb{P}}
\newcommand{\N}{\mathbb{N}}
\newcommand{\Z}{\mathbb{Z}}
\newcommand{\Q}{\mathbb{Q}}
\newcommand{\D}{\mathbb{D}}
\newcommand{\R}{\mathbb{R}}
\newcommand{\C}{\mathbb{C}}
\newcommand{\F}{\mathbb{F}}
\newcommand{\B}{\mathbb{B}}
\newcommand{\Norm}[2][]{\text{Norm}_{#1}(#2)}
\newcommand{\inner}[2]{\left\langle #1,#2 \right\rangle}
\newcommand{\floor}[1]{\lfloor #1 \rfloor}
\newcommand{\ceil}[1]{\lceil #1 \rceil}
\newcommand{\abs}[1]{| #1 |}
\newcommand{\card}[1]{| #1 |}
\newcommand{\curt}[1]{\sqrt[3]{#1}}
\newcommand{\Ker}[1]{\text{Ker}(#1)}
\newcommand{\Image}[1]{\text{Im}(#1)}
\newcommand{\Trace}[1]{\text{Tr}(#1)}
\newcommand{\Det}[1]{\text{Det}(#1)}
\newcommand{\degree}[1]{\partial #1}
\newcommand{\Pow}[1]{\mathcal{P}(#1)}


\title{Refresher Math Course}
\author{Paul Dubois}
\date{September 2021}



\begin{document}
	\maketitle
	
	\begin{abstract}
		This course teaches basic mathematical methodologies for proofs.
		It is intended for students with a lack of mathematical background, or with a lack of confidence in mathematics.
		The course will try to cover most of the prerequisites of the courses in the Master, mainly linear algebra, differential calculus, integration, and asymptotic analysis.
	\end{abstract}

	\tableofcontents
	\newpage
	
	\section*{Introduction}
	\paragraph{Presentation}
	\begin{itemize}
		\item Paul Dubois
		\item will be teaching this refresher math course
		\item email (for any question), answer within 1 working day
	\end{itemize}
	\paragraph{Course Format}
	\subparagraph{Lectures}
	\begin{itemize}
		\item 8*3h
		\item 1h20min lecture - $\nicefrac{1}{3}$h break - 1h20min lecture
		\item No pb class planned, but lectures will have integrated live exercises
		\item Interrupt if needed (but may also ask at the end of the lecture)
		\item Lectures are recorded (if ever needed)
		\item 1st lecture ever => too fast/too slow: let me know
		\item May assume you know a concept/notation that you have never heard of, let me know if this happens
	\end{itemize}
	\subparagraph{Examination}
	\begin{itemize}
		\item The course is pass/fail
		\item Most (in fact hopefully all) of you will pass
		\item There will be a full exercise sheet per lecture, it is advised to attempt it all (only one will be compulsory).
		\item Hand-in 1 exercise per lecture (i.e., 8 in total), due 2 weeks after the lecture
		\item Best $\nicefrac{(n-1)}{n}$ count (i.e., best $\nicefrac{7}{8}$ in our case), need avg $\geq 50 \%$ to pass
		\item In the unlikely event of not passing, will be able to do an extra work
	\end{itemize}
	\paragraph{Questions?}
	
	\newpage
	
	
	\chapter{Sets \& logic}

\section{Mathematical Objects \& Notations}
\subparagraph{Sets}
\begin{definition}[Sets]
	Unordered list of elements.
\end{definition}
\begin{notation}[Sets]
	$\in$, $\{ \text{True}, \text{False} \}$, $\{ a \mid condition \}$, $\{ a, b, c \dots \}$, $\emptyset$
\end{notation}
Need to be careful when defining set: some definitions are pathological.
\begin{remark}[Russell Paradox]
	Take $U = \{X \mid X \not\in X\}$.
	X in U => U not in U, U is a set, so not all sets are in U
	X not in U => X is a set
\end{remark}
\begin{notation}[Usual Sets]
	$\B$, $\N$, $\Z$, $\Q$, $\R$, $\C$, $\N^*$, $\R^+$...
\end{notation}

\subparagraph{Functions}
\begin{definition}[Functions]
	Assignment for a set to another.
\end{definition}
\begin{notation}[Function]
	$f: X \to Y$, $f(x)=blah$, $f: x \mapsto blah$.
\end{notation}
\begin{definition}[Predicate]
	Function to $\B$
\end{definition}
\begin{question}
	Which ones of these function are well-defined ?
	\begin{itemize}
		\item $f:k\in\{0,1,2,3,4\}\mapsto 24/k\in \N$
		\item $f:k\in \{1,2,3,4\}\mapsto 24/k\in \N$
		\item $f:k\in \{1,2,3,4,5\}\mapsto 24/k\in \N$
		\item $f:k\in \{1,2,3,4\}\mapsto k\in \{1,2\}$
		\item $f:k\in \{1,2,3,4\}\mapsto k\in \{1,2,3,4,5\}$
	\end{itemize}
\end{question}

\subparagraph{Quantifiers}
\begin{notation}[$\forall$]
	For all elements in set, e.g.: $\forall x \in \R, x^2 \geq 0$.
\end{notation}
\begin{notation}[$\exists$]
	There exists an element in set, e.g.: $\exists x \in \R \text{ s.t. } x^2 > 1$.
\end{notation}
\begin{notation}[$\exists !$]
	There exists a unique element in set, e.g.: $\exists ! x \in \R \text{ s.t. } x^2 \leq 0$.
\end{notation}
\begin{definition}[Subset / Inclusion]
	$X \subseteq Y$ if $\forall x \in X, x \in Y$
\end{definition}
\begin{definition}[Disjoint Sets]
	$X$ and $Y$ are disjoint if $\forall x \in X, x \not\in Y$ (or if $\forall y \in Y, y \not\in X$).
\end{definition}

\begin{definition}[Powerset]
	$\Pow{X} = \{ Y \mid Y \subseteq X \}$\\
	e.g.: $\Pow{\{1,2,3\}}=\{ \emptyset, \{1\},\{2\},\{3\}, \{1,2\},\{1,3\},\{2,3\}, \{1,2,3\} \}$
\end{definition}
\begin{definition}[Cartesian Product]
	$X \times Y = \{ (x,y) \mid x \in X, y \in Y \}$\\
	e.g.: $\{a,b\} \times \{1,2,3\} = \{ (a,1),(a,2),(a,3), (b,1),(b,2),(b,3) \}$\\
	Extension: $X_1 \times \dots \times X_n = \prod_{k=1}^n X_k$
\end{definition}



\section{Boolean algebra}
\subparagraph{Basic operators}
\begin{definition}[Conjonction]
	$x \land y = xy$
\end{definition}
\begin{definition}[Intersection]
	$X \cap Y = \{ z \mid (z \in X) \land (z \in Y) \}$
\end{definition}
\begin{remark}[Disjoint Sets and Intersection]
	Disjoint sets have empty intersection.
\end{remark}
\begin{definition}[Disjunction]
	$x \lor y = \min(x+y,1)$
\end{definition}
\begin{definition}[Union]
	$X \cup Y = \{ z \mid (z \in X) \lor (z \in Y) \}$
\end{definition}
\begin{definition}[Negation]
	$\lnot: 0,1 \mapsto 1,0$
\end{definition}
\begin{definition}[Set minus / Complement]
	$X \setminus Y = \{ x \in X \mid \lnot (x \in Y) \}$
\end{definition}
\begin{question}
	Selecting points outside a given region.
\end{question}
\subparagraph{Basic properties}
\begin{property}[Boolean algebra matching ordinary algebra]
	Same laws as ordinary algebra when one matches up $\lor$ with addition and $\land$ with multiplication.
	\begin{itemize}
		\item Associativity of $\lor$: $x \lor (y \lor z) = (x \lor y) \lor z$
		\item Associativity of $\land$: $x \land (y \land z) = (x \land y) \land z$
		\item Commutativity of $\lor$: $x \lor y  = y \lor x$
		\item Commutativity of $\land$: $x \land y  = y \land x$
		\item Distributivity of $\land$ over $\lor$:  $x \land (y \lor z) = (x \land y) \lor (x \land z)$
		\item $0$ is identity for $\lor$: $x \lor 0  = x$
		\item $1$ is identity for $\land$: $x \land 1  = x$
		\item $0$ is annihilator for $\land$: $x \land 0  = 0$
	\end{itemize}
\end{property}
\begin{property}[Boolean algebra specific properties]
	The following laws hold in Boolean algebra, but not in ordinary algebra: 
	\begin{itemize}
		\item Idempotence of $\lor$: $x \lor x = x$
		\item Idempotence of $\land$: $x \land x = x$
		\item Absorption of $\lor$ over $\land$: $x \lor (x \land y)  = x \land y$
		\item Absorption of $\land$ over $\lor$: $x \land (x \lor y)  = x \lor y$
		\item Distributivity of $\lor$ over $\land$:  $x \lor (y \land z) = (x \lor y) \land (x \lor z)$
		\item $1$ is annihilator for $\lor$: $x \lor 1 = 1$
	\end{itemize}
\end{property}
\begin{property}[De Morgan Laws]
	$\lnot (x \land y) = \lnot x \lor \lnot y$
	$\lnot (x \lor y) = \lnot x \land \lnot y$
\end{property}
\begin{proof}
	Truth-tables; prove De Morgan, others as exercise (or just believe me)
\end{proof}

\subparagraph{Other operators}
\begin{definition}[Exclusive Or]
	$x \oplus y$
\end{definition}
\begin{definition}[Implication]
	$x \implies y$
\end{definition}
\begin{property}[Implication and Inclusion]
	If $\forall x \in X, P_1(x) \implies P_2(x)$, then $\{ x \in X \mid P_1(x) \} \subset \{ x \in X \mid P_2(x) \}$.
\end{property}
\begin{proof}
	Trivial.
\end{proof}
\begin{definition}[If and only if]
	$x \iff y$
\end{definition}

\subparagraph{Negation of quantified propositions}
\begin{property}[Negation of $\forall$]
	$\mathrm{not}(\forall x\in X, P(x)) = \exists x\in X, \mathrm{not}(P(x))$
\end{property}
\begin{property}[Negation of $\exists$]
	$\mathrm{not}(\exists x\in X, P(x)) = \exists x\in X, \mathrm{not}(P(x))$
\end{property}
\begin{notation}[Quantifiers and the empty set]
	$\forall x \in \emptyset, \ \dots$ is true ;
	$\exists x \in \emptyset, \ \dots$ is false
\end{notation}



\section{Python}
=> use google colab'



	\chapter{Proofs methods}

\section{Direct implication}
Want to show $A$: may show $B$ and $B \implies A$, or $C$ and $C \implies B$ and $B \implies A$.

\section{Case dis-junction}
Split in cases.

E.g.: show $n$ and $n^2$ have the same parity (take $n$ odd then $n$ even).

\section{Contradiction}
Suppose the opposite, derive a contradiction (i.e. $A$ and $\not A$) and conclude.

E.g.: show $\sqrt{2} \not\in \Q$ (suppose $\sqrt{2}=\nicefrac{a}{b}$, WLOG $a,b \in \N$ co-prime).

\section{Induction}
Want to show $P_n$ for $n \geq n_0$: show $P_n \implies P_{n+1}$ and $P_{n_0}$.

E.g.: show $\sum_{k=0}^{n} k = \frac{n(n+1)}{2}$ for all $n \in \N$.

\section{Existence and Uniqueness}
It is common to show existence and/or uniqueness.

E.g.: Existence and uniqueness in Euclidean division: 
$$\forall a \in \Z, b \in \N^*, \exists ! \ q \in \Z, r \in \left[ 0, b \right[ \cap \N \text{ s.t. } a=bq+r$$
Use $q = \max\{ k \in \N \mid bk \leq a \}$, $r = a-bq$.

\begin{question}
	\begin{itemize}
		\item Show that $n$ divisible by 6 implies $n$ divisible by 2 and 3.
		\item Show $\sqrt{3} \not\in \Q$.
		\item Show that $11n-6$ is divisible by 5 for every positive integer $n$.
		\item Show that $2^n \geq 2n$ for all $n \in \N$
	\end{itemize}
\end{question}
	\chapter{Functions Properties}
$$f: X \to Y \quad A \subseteq X, B \subseteq Y$$
\begin{definition}[Image]
	$f(A) = \{ y \in Y \mid \exists x \in A \text{ s.t. } f(x)=y \}$
\end{definition}
\begin{definition}[Inverse Image]
	$f^{-1}(B) = \{ x \in X \mid f(x) \in B \}$
\end{definition}
\begin{definition}[Fiber]
	Fiber of $y$ is inverse image of $\{y\}$.
\end{definition}
\begin{definition}[Well definedness]
	$\forall x \in X, \exists ! y \in Y \text{ s.t. } f(x) = y$\\
	%	-- or --\\
	%	$\forall x \in X, \exists y \in Y \text{ s.t. } f(x) = y$\\
	%	$\forall y,y' \in Y, y \neq y', f^{-1}(y) \cap f^{-1}(y') = \emptyset$
\end{definition}
\begin{definition}[Injectivity]
	$\forall x,x' \in X, x \neq x', f(x) \neq f(x')$
\end{definition}
\begin{definition}[Surjectivity]
	$\forall y \in Y, \exists x \in X \text{ s.t. } f(x) = y$
\end{definition}
\begin{definition}[Bijectivity]
	Injectivity plus Surjectivity:
	$\forall y \in Y, \exists! x \in X \text{ s.t. } f(x) = y$
\end{definition}
\begin{definition}[Invertibility]
	$f^{-1}: Y \to X$ well defined.
\end{definition}
\begin{remark}[Alternative Definition of Inverse]
	$f \circ f^{-1} = Id \mid_X$  and $f^{-1} \circ f = Id \mid_Y$
\end{remark}
\begin{remark}[Invertibility and Bijectivity]
	$f$ bijective $\iff$ $f$ invertible.
\end{remark}
\begin{remark}[Inverse is Invertible]
	$f^{-1}$ is invertible, and $(f^{-1})^{-1}=f$.
\end{remark}
\begin{property}[Injections between finite intervals]
	$m,n \in \N^*$, there exists an injection $f:\llbracket 1;m \rrbracket \rightarrow \llbracket 1;n \rrbracket$ if and only if $m \leq n$.
\end{property}
\begin{proof}
	By induction on $m$, carefully checking $m \leq n$.
\end{proof}
\begin{property}[Bijections between finite intervals]
	$n,m \in \N^*$, there exists a bijection $f:\llbracket 1;m \rrbracket \rightarrow \llbracket 1;n \rrbracket$ if and only if $m=n$.
\end{property}
\begin{proof}
	Use last property \& inverse.
\end{proof}
\begin{property}[Compositions]
	Composition preserve injectivity/surjectivity/bijectivity/invertibility:
	$f: X \to Y, \ g: Y \to Z \text{ injectives } \implies f \circ g \text{ is injective}$\\
	$f: X \to Y, \ g: Y \to Z \text{ surjectives } \implies f \circ g \text{ is surjective}$\\
	$f: X \to Y, \ g: Y \to Z \text{ bijectives/invertibles } \implies f \circ g \text{ is bijective/invertible}$
\end{property}
\begin{proof}
	Trivial.
\end{proof}
\begin{property}
	An injection between two sets of the same size is bijective.
\end{property}
\begin{proof}
	By contradiction.
\end{proof}


	\chapter{Finite Cardinalities}

\begin{definition}[Cardinality] For finite sets:\\
	\emph{\underline{Intuitively}:} $\card{X} = n \in \N$ if there are $n$ elements in the set.\\
	\emph{\underline{Mathematically}:} $\card{X} = n \in \N$ if there is a bijection between $X$ and $\llbracket 1,n \rrbracket$.
\end{definition}
\begin{property}[Cardinality of Disjoints]
	$X,Y$ disjoint sets: $\card{X \cup Y} = \card{X} + \card{Y}$\\
	Extension: $X_1, \dots, X_n$ pairwise disjoint sets (i.e. $X_i \cap X_j = \emptyset \ \forall i \neq j$): $\card{\bigcup_{k=1}^n X_k} = \sum_{k=1}^{n} \card{X_k}$
\end{property}
\begin{proof}
	Shift bijection of $Y$ by $\card{Y}$; use induction.
\end{proof}
\begin{property}[Cardinality of Complement]
	$X \subseteq Y$: $\card{Y \setminus X} = \card{Y} - \card{X}$
\end{property}
\begin{proof}
	Use previous property with $X$ \& $Y \setminus X$ disjoint.
\end{proof}
\begin{property}[Cardinality of Cartesian Products]
	$X,Y$ sets: $\card{X \times Y} = \card{X} * \card{Y}$\\
	Extension: $X_1, \dots, X_n$ sets: $\card{\prod_{k=1}^n X_k} = \prod_{k=1}^{n} \card{X_k}$
	%proof: 
\end{property}
\begin{proof}
	$X \times \{y_k\}$  are all disjoint for $k \in \llbracket 1,\card{Y} \rrbracket$; use induction.
\end{proof}
\begin{property}[Cardinality of Sub-list]
	$X$ sets: $\card{ \{ Y \text{ list} \mid \card{Y}=n \text{ and } y \in Y \implies y \in X \} } = \card{X}^n$
\end{property}
\begin{proof}
	Just count!
\end{proof}
\begin{property}[Cardinality of Ordered Subsets]
	$X$ sets: $\card{ \{ Y \text{ ordered set} \mid \card{Y}=n \text{ and } y \in Y \implies y \in X \} } = \frac{\card{X}!}{(\card{x}-n)!}$
\end{property}
\begin{proof}
Just count!
\end{proof}
\begin{property}[Cardinality of Subsets]
	$X$ sets: $\card{ \{ Y \subseteq X \mid \card{Y}=n \} } = \binom{\card{X}}{n}$
\end{property}
\begin{proof}
	Just count!
\end{proof}
\begin{property}[Cardinality of Sets of Functions]
	%Similar to Cartesian product:\\
	$\card{ \{f: X \to Y\} } = \card{Y}^{\card{X}}$
\end{property}
\begin{proof}
	Just count!
\end{proof}
\begin{property}[Cardinality of Sets of Injections]
	$\card{ \{f: X \to Y \mid f \text{ injective} \} } = \frac{\card{Y}!}{(\card{Y}-\card{X})!}$
\end{property}
\begin{proof}
	Count (without repetition).
\end{proof}
\begin{property}[Cardinality of Sets of Surjections]
	$\card{ \{f: X \to Y \mid f \text{ surjective} \} } = \card{Y}^{\card{X}} - \card{Y}*(\card{Y}-1)^{\card{X}}$
\end{property}
\begin{proof}
	All functions but the non surjective ones.
\end{proof}
\begin{property}[Cardinality of Sets of Bijections]
	$\card{ \{f: X \to Y \mid f \text{ bijective} \} } = \card{Y}! = \card{X}!$
\end{property}
\begin{proof}
	Bijection is an injection between two sets of the same size.
\end{proof}

\begin{question}
	\begin{itemize}
		\item For $n$ students, if we record the order of people getting out of the room, how many possibilities are there?
		\item Bench for 10 people, we have 5 boys, 5 girls, how many arrangements are there such that two boys/two girls are never seated next to each others?
		\item Bench for 11 people, we have 6 boys, 5 girls, how many arrangements are there such that two boys/two girls are never seated next to each others?
	\end{itemize}
\end{question}



	\chapter{Infinite Cardinalities}
\begin{definition}[Alphabet]
	$\mathcal{A} = \{ a,b,c, \dots, z \}$
\end{definition}
To compare the size of infinite sets, we use bijections, injections:
\begin{definition}[Comparing Sets]
	$f: X \to Y \text{ injective} \implies \card{X} \leq \card{Y}$
	$f: X \to Y \text{ surjective} \implies \card{X} \geq \card{Y}$
	$f: X \to Y \text{ bijective} \implies \card{X} = \card{Y}$
\end{definition}
Note that together with $\card{\left[ 1,n \right]}=n$, this defines cardinality.
\begin{definition}[Countable sets]
	A set is countable if it has the same cardinality as the naturals (i.e. $X$ is countable if $\card{X} = \card{\N}$).
\end{definition}

\begin{property}[Countable Union Finite]
	$\card{\N \cup \mathcal{A}} = \card{\N}$
\end{property}
\begin{property}[Countable Union Countable / Integers]
	$\card{\Z} = \card{\N \cup \N^*} = \card{\N}$
\end{property}
\begin{property}[Countable Union of Finites]
	$\card{X_n}<\infty \ \forall n \in \N \implies \card{\bigcup_{n \in \N} X_n} = \card{\N}$
	%proof: stack the X_n on \N
\end{property}
\begin{property}[Countable Union of Countables / Rationals]
	$\card{\Q} = \card{\bigcup_{n \in \N^*} \{ \nicefrac{m}{n} \mid m \in \Z \}} = \card{\N}$
\end{property}
\begin{property}[Power set of Countables / Reals]
	$\card{ \left[ 0,1 \right[ } = \card{\Pow{\N}} = \card{\{0,1\}^{\N}} > \card{\N}$
\end{property}
\begin{property}[Bounded \& Unbounded Reals]
	$\card{ \left[ 0,1 \right[ } = \card{ \R }$
\end{property}
\begin{property}[Reals and Product of Reals]
	$\card{ \left[ 0,1 \right[ } = \card{ \left[ 0,1 \right[^2 }$
\end{property}

\begin{question}
	\begin{itemize}
		\item What is $\N \times \N \times \N$ compared to $\N$?
		\item What is $\N \times \N \times \N \times \N$ compared to $\N$?
		\item What is $\R \times \R$ compared to $\R$?
	\end{itemize}
\end{question}




	\chapter{Spaces}
Mathematical Space: Object based on a set with more structure.
\section{Metric Space}
A metric space is a set $X$ together with a metric distance $d: X \times X \to \R^+$.\\
$d$ is a metric if it satisfies the following axioms:
\begin{itemize}
	\item Non-degenerative: $d(x,y)=0 \iff x=y$
	\item Symmetric: $d(x,y) = d(y,x)$
	\item Triangle inequality: $d(x,z) \leq d(x,y) + d(y,z)$
\end{itemize}

\section{Norm Space}
A norm space is a set $X$ together with a norm $\abs{\_}: X \to \R^+$.\\
$\abs{\_}$ is a norm if it satisfies the following axioms:
\begin{itemize}
	\item Non-degenerative: $\abs{x}=0 \iff x=0$
	\item Homogeneity: $\abs{\lambda x} = \lambda \abs{x} \qquad \lambda \in \R^+$
	\item Triangle inequality: $\abs{x+y} \leq \abs{x} + \abs{y}$
\end{itemize}
\begin{property}[Norm Implies Metric]
	Letting $d(x,y) = \abs{x-y}$.
\end{property}

\section{Inner Product Space}
An inner product space is a set $X$ together with an inner product $\inner{\_}{\_}: X \times X \to \C$.\\
$\inner{\_}{\_}$ is an inner product if it satisfies the following axioms:
\begin{itemize}
	\item Linear (in $1^{\text{st}}$ argument): $\inner{\lambda x}{y} = \lambda \inner{x}{y} \quad \lambda \in \C$ and $\inner{x+x'}{y} = \inner{x}{y} +\inner{x'}{y}$
	\item Conjugate symmetry: $\abs{x+y} \leq \abs{x} + \abs{y}$
	\item Positive definiteness $\inner{x}{x}>0 \ \forall x \neq 0$
	\item \textit{(implied)} Non-degenerative: $\inner{x}{0}=0$ and $\inner{0}{x}=0$
	\item \textit{(implied)} Conjugate linear (in $2^{\text{nd}}$ argument): $\inner{x}{\lambda y} = \bar{\lambda} \inner{x}{y} \quad \lambda \in \C$ and $\inner{x}{y+y'} = \inner{x}{y} +\inner{x}{y'}$
\end{itemize}
\begin{property}[Inner Product implies Norm]
	Letting $\abs{x} = \sqrt{\inner{x}{x}}$.
\end{property}

\begin{property}[Cauchy-Schwarz inequality]
	$\inner{x}{y}^2 \leq \inner{x}{x} \inner{y}{y}$
\end{property}
\begin{proof}
	Let $P(\lambda)=\inner{x+\lambda y}{x+\lambda y}$.
	This polynomial is never negative, so its discriminant must be non-positive.
	Deduce the inequality from $\Delta \geq 0$.
\end{proof}
\begin{definition}[Orthogonal / Normal]
	$x,y \text{ orthogonal } \iff x \perp y \iff \inner{x}{y}=0$
\end{definition}
\begin{property}[Pythagoras Theorem]
	$x \perp y \implies \abs{x+y}^2 = \abs{x}^2 + \abs{y}^2$
\end{property}
\begin{property}[Parallelogram Identity]
	$\abs{x+y}^2 +\abs{x-y}^2 = 2 (\abs{x}^2 + \abs{y}^2)$
\end{property}
\begin{property}[Polarization Identity]
	$4 \inner{x}{y} = \abs{x+y}^2 - \abs{x-y}^2 + i(\abs{x+iy}^2 - \abs{x-iy}^2)$
\end{property}

\begin{question}
	Draw ball of radius one in $\R^2$ for the following norms: $\abs{\_}_1$, $\abs{\_}_2$, $\abs{\_}_3$, $\abs{\_}_{\infty}$.
\end{question}



\section{Openness}
Here, we work over $(X,d)$, a metric space.
\begin{definition}[Open Ball]
	$B(x_0,r) = \{ x \in X \mid d(x,x_0)<r \}$
\end{definition}
\begin{definition}[Closed Ball]
	$\overline{B}(x_0,r) = \{ x \in X \mid d(x,x_0) \leq r \}$
\end{definition}
\begin{definition}[Open Set]
	$U \text{ is open } \iff \forall x \in U, \exists \epsilon>0 \text{ s.t. } B(x_0,\epsilon) \subseteq U$
\end{definition}
\begin{definition}[Closed Set]
	$C \text{ is closed } \iff X \setminus C \text{ is open}$
\end{definition}
\begin{property}
	Open balls are open.
\end{property}
\begin{proof}
	Use triangle inequality \& draw scheme
\end{proof}
\begin{property}
	Closed balls are closed.
\end{property}
\begin{proof}
	Use triangle inequality \& draw scheme
\end{proof}


	\chapter{Limit Behaviors}

%\begin{definition}[Sequence]
%	$(x_n)_{n\in \N}$ is a sequence in $\F$ if $x_n \in \F \ \forall n \in \N$
%\end{definition}

\section{Convergence \& Divergence}

\begin{definition}[$(x_n) \subseteq \F$ converges to $x \in \F$]
	$\,\forall \varepsilon > 0, \ \exists N \text{ s.t. } \forall n \geq N, \ d(x_n,x) < \varepsilon$
\end{definition}
We write $x_n \xrightarrow[n \to +\infty]{} x \text{ or } \lim\limits_{n \to +\infty} x_n = x$.
Note that convergence is defined w.r.t. a metric (or a norm/inner product, which induces a metric).

\begin{definition}[$(x_n) \subseteq \R$ diverges to $+\infty$]
	$\,\forall M \in \R, \ \exists N \text{ s.t. } \forall n \geq N, \ x_n > M$
\end{definition}
We write $x_n \xrightarrow[n \to \infty]{} +\infty \text{ or } \lim\limits_{n \to +\infty} x_n = +\infty$.
Note that divergence is only defined over $\R$; divergence to $-\infty$ is defined similarly.

\begin{definition}[Sub-sequence]
	$\phi: \N \to \N$ strictly increasing defines the sub-sequence $(x_{\phi(n)})$ of the sequence $(x_n)$.
\end{definition}

\begin{property}[Convergence \& Divergence of Sub-sequences]
	$x_n \to x \implies x_{\phi(n)} \to x$ \\
	moreover, $x_n \to +\infty \implies x_{\phi(n)} \to +\infty$
\end{property}

\begin{definition}[$f:X \to Y$ converges to $y \in Y$ at $x \in X$]
	$\forall \epsilon>0, \ \exists \delta>0 \text{ s.t. } d_X(x,x')<\delta \implies d_Y(y,y')<\epsilon$
\end{definition}
Equivalent definition: $\forall x_n \to x \text{ as } n \to +\infty, y_n = f(x_n) \to y$; we write $\lim\limits_{x' \to x} f(x') = y$

\begin{question}
	\begin{itemize}
		\item $\lim\limits_{x \to a} \phi(f(x))$
		\item $\lim\limits_{x \to a} f(x)+g(x)$
		\item $\lim\limits_{x \to a} f(x)*g(x)$
		\item $\lim\limits_{x \to a} f(x)/g(x) \qquad g(x) \neq 0$
	\end{itemize}
\end{question}
\begin{proof}
	left as exercise
\end{proof}

\begin{property}["Determinate Forms"]
	$\frac{1}{0} = \infty$, $\frac{1}{\infty} = 0$
\end{property}
\begin{property}["Indeterminate Forms"]
	$\frac{0}{0}$, $\frac{\infty}{\infty}$, $0 \times \infty$, $1^\infty$, $\infty - \infty$, $0^0$, $\infty^0$
\end{property}
E.g.: $x^2 \times \frac{1}{x} \to \infty$; $x^2 \times \frac{1}{x^2} \to 1$; $x^2 \times \frac{1}{x^3} \to 0$.

\begin{theorem}[Fixed Point Theorem]
	$x_{n+1} = f(x_n) \text{ and } (x_n) \to l \implies l = f(l)$ (i.e. $l$ is a fixed point of $f$). 
\end{theorem}
\begin{proof}
	easy: $x_n$ and $x_{n+1}$ must both go to $l$
\end{proof}
E.g.: $x_0=1$ and $x_{n+1}=\frac{x_n^2+2}{2x_n}$ give $x_n \to \sqrt{2}$.


\section{Maximum vs Supremum}

\begin{definition}[$a$ maximum of $A$]
	$a \in A$ and $\forall x \in A, x \leq a$
\end{definition}
Maximum doesn't always exists (even if $A$ is bounded).
\begin{definition}[$a$ supremum of $A$]
	$\forall \epsilon>0, \exists \, x \in A \text{ s.t. } a-\epsilon \leq x \leq a$
\end{definition}
Supremum is the "smallest upper bound". Exists if $A$ is bounded.

\begin{question}
	Max, sup of $\left[ 0,1 \right]$, $\left[ 0,1 \right[$, $\R^+$.
\end{question}

\begin{remark}
	Can define minimum \& infimum similarly
\end{remark}

\begin{theorem}[Extremum \& Convergence]
	$(x_n) \subseteq \R$ increasing:
	\begin{itemize}
		\item if $(x_n)$ is upper-bounded, then $\lim\limits_{n \to +\infty} x_n = \sup \left\{ x_n \mid n \in \N \right\}$
		\item else, $\lim\limits_{n \to +\infty} x_n = +\infty$
	\end{itemize}
\end{theorem}
\begin{proof}
	easy (by cases)
\end{proof}



\section{Continuity}

\begin{definition}[$f$ continuous at $x$]
	$\lim\limits_{x' \to x} f(x') = f(x)$
\end{definition}
\begin{definition}[$f$ continuous on $X$]
	$\forall x \in X, f \text{ continuous } x$
\end{definition}
\begin{question}
	Show $x^n$ is continuous (for all $n$).
\end{question}
"can be plotted in a single trace/line; without lifting the pen"
[Lipschitz-continuous??]



\section{Asymptotic Analysis}

\begin{definition}[Asymptote]
	"A curve is a line such that the distance between the curve and the line approaches zero as one or both of the $x$ or $y$ coordinates tends to infinity." \\
	i.e. $\lim\limits_{x \to \infty} f(x)-l(x) = 0$ (in the case of $x \to \infty$).
\end{definition}
\paragraph{Horizontal Asymptote}
E.g.: $f(x) = \frac{x+1}{x}$ (asymptote is $y=1$ as $x \to \infty$).
\paragraph{Vertical Asymptote}
E.g.: $g(x) = \frac{1}{x-2}$ (asymptote is $x=2$ as $y \to \infty$).
\paragraph{Oblique Asymptote}
E.g.: $h(x) = \frac{3x^2+2x+1}{x}$ (asymptote is $y=3x+2$ as $x,y \to \infty$).



\section{Series}
Joke: I once asked someone out to "checkout some series", they went home disappointed... still don't know why.
\begin{definition}[Series]
	A series is a sequence $(S_n)$ with general term $x_n$ defined by $S_n = \sum_{k=0}^{n} x_k$.\\
	It is alternating if $x_k x_{k+1} < 0 \ \forall k \in \N$.
\end{definition}
\begin{definition}[Series Convergence]
	The series $(S_n)$ converges if $\left( \sum_{k=0}^{n} x_k \right)$ converges as a sequence.
	The series $(S_n)$ converges absolutely if $\left( \sum_{k=0}^{n} \abs{x_k} \right)$ converges as a sequence.
\end{definition}
E.g.: $\sum_{k=1}^n 1$ is obviously divergent;
$\sum_{k=1}^n \frac{1}{2^k}$ is convergent (to 1).

\begin{property}
	$\sum_{k=0}^n a^k$ is:
	\begin{itemize}
		\item Absolutely convergent for $\abs{a} < 1$, converging to $\frac{1}{1-a}$.
		\item Divergent for $\abs{a} \geq 1$, bounded for $a=-1$, unbounded else.
	\end{itemize}
\end{property}
\begin{proof}
	easy (sum of geometric series)
\end{proof}

\begin{property}[Necessary Condition for Convergence of Series]
	If $(S_n)$ converges, then $x_n \to 0$.
\end{property}
\begin{proof}
	trivial (by contradiction)
\end{proof}
However, his is \textbf{not} a sufficient condition: $\sum_{k=1}^n \frac{1}{k}$ is a counter-example.

\begin{property}[Criterion for Convergence of Alternating Series]
	If $\sum_{n \in \N} x_n$ is alternating, $(\abs{x_n})$ is decreasing, and $\lim\limits_{n \to \infty} x_n = 0$, then $\sum_{n \in \N} x_n$ converges.
\end{property}
\begin{proof}
	WLOG, $x_{2n}>0$ and $x_{2n-2}<0$:
	$\sum_{k=0}^{2n} x_k$ is increasing, and upper bounded by $x_0+x_1$, therefore converges;
	similarly, $\sum_{k=0}^{2n+1} x_k$ is decreasing, and lower bounded by $x_0$, therefore converges as well.
	$\sum_{k=0}^{2n} x_k$ and $\sum_{k=0}^{2n+1} x_k$ must have the same limit as $\sum_{k=0}^{2n+1} x_k - \sum_{k=0}^{2n} x_k = x_{2n+1} \to 0$.
	Thus, $\sum_{k=0}^{2n} x_k$ must be convergent.
\end{proof}

\begin{property}[Comparison Test for Convergence of Series]
	$\forall n \geq n_0, 0 \leq a_n \leq b_n$:
	\begin{itemize}
		\item If $\sum_{n \in \N} b_n$ converges, then $\sum_{n \in \N} a_n$ converges as well.
		\item If $\sum_{n \in \N} a_n$ diverges, then $\sum_{n \in \N} b_n$ diverges as well.
	\end{itemize}
\end{property}
\begin{proof}
	easy by def
\end{proof}
E.g.: $\sum_{n \geq 2} \frac{1}{n^2}
\leq \sum_{n \geq 2} \frac{1}{n(n+1)}
= \sum_{n \geq 2} \frac{1}{n}-\frac{1}{n+1}
= \frac{1}{2} < \infty$

\begin{property}[Integration Test for Convergence of Series]
	$\sum_{n \in \N} f(n) \leq \int_{x=0}{\infty} f(x)$\\
	So if $\int_{x=0}{\infty} f(x) < \infty$, and $f(x)$ is decreasing, then $\sum_{n \in \N} f(n)$ converges.
\end{property}
\begin{proof}
	easy by def
\end{proof}
E.g.: $\sum_{n \geq 2} \frac{1}{n^2}
\leq \int_{x = 2}^{\infty} \frac{1}{x^2}
= \left[ -\frac{1}{x} \right]_{x=2}^{x=\infty}
= -0 +\frac{1}{2} < \infty$




	
	
	
\end{document}